\chapter{Generalized linear models and the exponential family}
\section{问题}
\begin{enumerate}
\item 有什么性质
\item 充分统计是什么
\item 什么是广义线性模型
\item 链接函数与激活函数
\item 充分统计与矩的关系
\item 伯努利与logistic回归的关系
\end{enumerate}
\section{指数族}
\begin{enumerate}
\item 有限充分统计
\item 具有共轭分布
\item 广义线性模型的核心
\item 变分的核心
\end{enumerate}
\subsection{定义}
\begin{equation}
p(x|\theta) = \frac{1}{Z(\theta)}
h(x)exp(\theta^T\phi(x))
\end{equation}

\begin{equation}
p(x|\theta) =
h(x)exp(\theta^T\phi(x)-A(\theta))
\end{equation}

\subsection{对数配分函数}

\begin{equation}
\frac{dA}{d\theta} = E[\phi(x)]
\end{equation}

\begin{equation}
\frac{d^2A}{d\theta^2} = var[\phi(x)]
\end{equation}

\begin{equation}
\frac{\partial^2A}{\partial\theta_i\partial\theta_j}=
E[\phi_i(x)\phi_j(x)] - E[\phi_i(x)]E[\phi_j(x)]
\end{equation}

\begin{equation}
\nabla A(\theta) = cov[\phi(x)]
\end{equation}

\section{sigmoid函数}
sigmoid函数可以通过伯努利分布链接函数的逆函数推导出来,
也可以通过产生式模型推导出来.

\section{公式推导}
\begin{enumerate}
\item 伯努利分布指数族推导
\item 离散分布指数族推导
\item 高斯分布指数族推导
\item 指数族充分统计推导
\item 指数族均值推导
\item 指数族矩推导
\end{enumerate}

